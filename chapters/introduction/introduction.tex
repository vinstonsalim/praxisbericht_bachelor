\chapter{Introduction}
\label{ch:introduction}


\section{History and Background of Local Brand X}
\label{sec:intro:background}
Local Brand X, an innovative company founded in 2018 and based in Mainz, specializes in marketing management software. The company focuses on industries such as insurance, franchises, chain stores, and retail. Local Brand X has developed the Local Marketing Platform, a tool designed to assist businesses in supporting their sales or distribution partners with the creation and execution of marketing initiatives throughout the customer journey. The primary objective of this platform is to ensure consistent local marketing, enhance local visibility, and thereby strengthen the brand. Currently, over 50,000 users from prominent brands such as Audio Service, BayWa, BBBank, and HanseMerkur utilize solutions provided by Local Brand X.

The Local Marketing Platform facilitates the planning, creation, and implementation of local marketing measures and campaigns by sales partners. It supports cross-media local marketing, including online, social media, print, and video marketing, enabling a diverse range of advertising activities with minimal prior knowledge required. The platform ensures seamless compliance with corporate design and allows for uniform campaign and budget planning across the company and brand. Automated processing of local marketing measures frees up time and resources at the head office for strategic planning and design, enhancing overall company awareness and success. The Local Marketing Platform's modular structure and the expansion options provided by Local Brand X add-ons enable it to adapt to varying needs and marketing priorities.

%
% Section: Motivation
%
\section{Problem Statement}
\label{sec:intro:motivation}
In today's digital era, users engage with an extensive array of web applications and services to accomplish various tasks. This interaction often requires navigating through multiple platforms, each with its own unique interface and login requirements, resulting in fragmented workflows and decreased productivity. The need to switch between different tabs and applications not only consumes time but also increases cognitive load, ultimately impacting user efficiency and experience.

Integrating data from one service to another service could be a severe challange, and time consuming, hence a manual input of the data could lead to a potential error. Creating a website than can handle the integration will just create another additional problem to the user, since the user will have to switch between the website and manually input the data to the service.

% This is where the part, where the browser extension comes to play.
% TODO: FINISH THIS
While numerous browser extensions are available, they typically address specific functionalities in isolation, such as bookmarking, password management, or content filtering. There is a notable absence of a unified tool that integrates these essential functions into a single, coherent interface. This lack of integration presents significant challenges in managing multiple online activities, leading to inefficiencies and complexities that hinder user productivity. Apart from that integrating the data from one service to another service can be a severe challenges

%
% Section: Ziele
%
\section{Purpose of the Work}
\label{sec:intro:goal}

The objective of this project is to develop a web browser extension that serves as a comprehensive one-stop solution. This extension aims to consolidate various functionalities, streamline workflows, and simplify the management of online activities. By providing a unified interface, the proposed solution will enhance user productivity, reduce cognitive load, and improve the overall user experience. Additionally, the development will prioritize user-friendly design, robust security features, and compatibility across different web browsers to ensure broad accessibility and utility.

\subsection{Functional Criteria}
The browser extension will incorporate several key functionalities:

\begin{itemize}
    \item \textbf{Bookmark Management:}
    The web browser extension will include a robust bookmark management feature that allows users to easily save, organize, and access specific marketing platform web pages. This functionality will help the Customer Success Department quickly access relevant information and resources, enhancing their efficiency and productivity. It ensures that all users who install the extension have access to the same bookmarks without interfering with their existing browser bookmarks. This eliminates the need for employees to sign in with the same Google account (as Google Chrome is the preferred browser at Local Brand X) to access shared bookmarks.

    \item \textbf{Synchronization from ClickUp to MOCO:}
    The integration of ClickUp with MOCO will enable seamless synchronization of tasks between the project management tool and the ERP system. Tasks created or updated in ClickUp will automatically reflect in MOCO, ensuring that project-related activities are aligned with business processes. This synchronization will minimize the need for manual data entry, reduce errors, and ensure that all team members have access to the latest information, enhancing overall project coordination and efficiency. MOCO will manage accounting, payment invoices for customers, employee salaries, and paychecks directly, as it integrates with other tools like DATEV and ClickUp.

    \item \textbf{Integration of Sentry's Issues Transfer to ClickUp:}
    This feature will facilitate the automatic transfer of issues and error reports from Sentry, a monitoring tool, to ClickUp. When Sentry detects an issue, it will automatically log it as a task in ClickUp, where it can be tracked and managed by the development team. This integration will streamline the error management process, ensure timely resolution of issues, and improve the overall quality and reliability of the software products. It will help the development team quickly identify and resolve issues, reducing downtime and enhancing the user experience, as Sentry provides detailed information about errors, aiding in the rapid identification of root causes.
\end{itemize}

\subsection{Non-Functional Criteria}
The development of the web browser extension will also focus on several non-functional criteria to ensure its effectiveness and usability:

\begin{itemize}
    \item \textbf{User-Friendly Design:}
    The extension will feature an intuitive and easy-to-navigate interface that allows users to access and utilize its functionalities with minimal effort. The design will prioritize simplicity and efficiency, enhancing the overall user experience.

    \item \textbf{Robust Security Features:}
    Security will be a primary consideration in the development of the extension. It will include measures such as data encryption, secure authentication, and protection against common web vulnerabilities to safeguard user information and ensure privacy.

    \item \textbf{Compatibility Across Different Web Browsers:}
    To ensure broad accessibility and utility, the extension will be developed to be compatible with multiple web browsers. This will allow users to benefit from the extension's features regardless of their preferred browsing platform.

    \item \textbf{Performance and Reliability:}
    The extension will be optimized for performance, ensuring that it runs smoothly without significantly impacting the browser’s speed or responsiveness. Additionally, it will be designed for reliability, minimizing crashes and ensuring consistent operation.
\end{itemize}